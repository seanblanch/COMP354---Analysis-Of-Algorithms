%
% 354 Homework 3-2
%
\documentclass[12pt,twoside]{article}

% Cross-references for handout numbers.

% Updated to include SMA course for Fall 2001 -- cel

\newcommand{\name}{}


\usepackage{latexsym}
%\usepackage{bbm}
\usepackage{times,url}
\usepackage{clrscode3e}

\newcommand{\csucist}[1]{\begin{description}
\item[CSUCI students:] #1
\end{description}}

\newcommand{\profs}{Ryan McIntyre}
\newcommand{\subj}{354}

\newlength{\toppush}
\setlength{\toppush}{2\headheight}
\addtolength{\toppush}{\headsep}

\newcommand{\htitle}[2]{\noindent\vspace*{-\toppush}\newline\parbox{6.5in}
{\textit{Analysis of Algorithms: 354}\hfill\name\newline
CSU Channel Islands \hfill #2\newline
\profs\hfill #1 \vspace*{-.5ex}\newline
\mbox{}\hrulefill\mbox{}}\vspace*{1ex}\mbox{}\newline
\begin{center}{\Large\bf #1}\end{center}}

\newcommand{\handout}[2]{\thispagestyle{empty}
 \markboth{#1}{#1}
 \pagestyle{myheadings}\htitle{#1}{#2}}

\newcommand{\htitlewithouttitle}[2]{\noindent\vspace*{-\toppush}\newline\parbox{6.5in}
{\textit{Analysis of Algorithms}\hfill#2\newline
CSU Channel Islands \hfill 354\newline
%Singapore-MIT Alliance \hfill SMA5503\newline
\profs\hfill Handout #1\vspace*{-.5ex}\newline
\mbox{}\hrulefill\mbox{}}\vspace*{1ex}\mbox{}\newline}

\newcommand{\handoutwithouttitle}[2]{\thispagestyle{empty}
 \markboth{Handout \protect\ref{#1}}{Handout \protect\ref{#1}}
 \pagestyle{myheadings}\htitlewithouttitle{\protect\ref{#1}}{#2}}

\newcommand{\exam}[2]{% parameters: exam name, date
 \thispagestyle{empty}
 \markboth{\subj\ #1\hspace{1in}Name\hrulefill\ \ }%
          {\subj\ #1\hspace{1in}Name\hrulefill\ \ }
 \pagestyle{myheadings}\examtitle{#1}{#2}
 \renewcommand{\theproblem}{Problem \arabic{problemnum}}
}
\newcommand{\examsolutions}[3]{% parameters: handout, exam name, date
 \thispagestyle{empty}
 \markboth{Handout \protect\ref{#1}: #2}{Handout \protect\ref{#1}: #2}
% \pagestyle{myheadings}\htitle{\protect\ref{#1}}{#2}{#3}
 \pagestyle{myheadings}\examsolutionstitle{\protect\ref{#1}} {#2}{#3}
 \renewcommand{\theproblem}{Problem \arabic{problemnum}}
}
\newcommand{\examsolutionstitle}[3]{\noindent\vspace*{-\toppush}\newline\parbox{6.5in}
{\textit{Analysis of Algorithms}\hfill#3\newline
CSU Channel Islands \hfill 354\newline
%Singapore-MIT Alliance \hfill SMA5503\newline
\profs\hfill Handout #1\vspace*{-.5ex}\newline
\mbox{}\hrulefill\mbox{}}\vspace*{1ex}\mbox{}\newline
\begin{center}{\Large\bf #2}\end{center}}

\newcommand{\takehomeexam}[2]{% parameters: exam name, date
 \thispagestyle{empty}
 \markboth{\subj\ #1\hfill}{\subj\ #1\hfill}
 \pagestyle{myheadings}\examtitle{#1}{#2}
 \renewcommand{\theproblem}{Problem \arabic{problemnum}}
}

\makeatletter
\newcommand{\exambooklet}[2]{% parameters: exam name, date
 \thispagestyle{empty}
 \markboth{\subj\ #1}{\subj\ #1}
 \pagestyle{myheadings}\examtitle{#1}{#2}
 \renewcommand{\theproblem}{Problem \arabic{problemnum}}
 \renewcommand{\problem}{\newpage
 \item \let\@currentlabel=\theproblem
 \markboth{\subj\ #1, \theproblem}{\subj\ #1, \theproblem}}
}
\makeatother


\newcommand{\examtitle}[2]{\noindent\vspace*{-\toppush}\newline\parbox{6.5in}
{\textit{Analysis of Algorithms}\hfill#2\newline
CSU Channel Islands \hfill 354 Fall 2020\newline
%Singapore-MIT Alliance \hfill SMA5503\newline
\profs\hfill #1\vspace*{-.5ex}\newline
\mbox{}\hrulefill\mbox{}}\vspace*{1ex}\mbox{}\newline
\begin{center}{\Large\bf #1}\end{center}}

\newcommand{\grader}[1]{\hspace{1cm}\textsf{\textbf{#1}}\hspace{1cm}}

%\newcommand{\points}[1]{[#1 points]\ }
\newcount\pointcount
\newcommand{\points}[1]{\pointcount=#1\relax \ifnum\pointcount=1 [#1 point]\else [#1 points]\fi\ }

\newcommand{\parts}[1]
{
  \ifnum#1=1
  (1 part)
  \else
  (#1 parts)
  \fi
  \ 
}

\newcommand{\bparts}{\begin{problemparts}}
\newcommand{\eparts}{\end{problemparts}}
\newcommand{\ppart}{\problempart}

%\newcommand{\lg} {lg\ }

\setlength{\oddsidemargin}{0pt}
\setlength{\evensidemargin}{0pt}
\setlength{\textwidth}{6.5in}
\setlength{\topmargin}{0in}
\setlength{\textheight}{8.5in}


\renewcommand{\cases}[1]{\left\{ \begin{array}{ll}#1\end{array}\right.}
\newcommand{\cif}[1]{\mbox{if $#1$}}
\newcommand{\cwhen}[1]{\mbox{when $#1$}}

\newcounter{problemnum}
\newcommand{\theproblem}{Problem \theproblemsetnum-\arabic{problemnum}}
\newenvironment{problems}{
        \begin{list}{{\bf \theproblem. \hspace*{0.5em}}}
        {\setlength{\leftmargin}{0em}
         \setlength{\rightmargin}{0em}
         \setlength{\labelwidth}{0em}
         \setlength{\labelsep}{0em}
         \usecounter{problemnum}}}{\end{list}}
\makeatletter
\newcommand{\problem}[1][{}]{\item \let\@currentlabel=\theproblem \textbf{#1}}
\makeatother

\newcounter{problempartnum}[problemnum]
\newenvironment{problemparts}{
        \begin{list}{{\bf (\alph{problempartnum})}}
        {\setlength{\leftmargin}{2.5em}
         \setlength{\rightmargin}{2.5em}
         \setlength{\labelsep}{0.5em}}}{\end{list}}
\newcommand{\problempart}{\addtocounter{problempartnum}{1}\item}

\newenvironment{truefalseproblemparts}{
        \begin{list}{{\bf (\alph{problempartnum})\ \ \ T\ \ F\hfil}}
        {\setlength{\leftmargin}{4.5em}
         \setlength{\rightmargin}{2.5em}
         \setlength{\labelsep}{0.5em}
         \setlength{\labelwidth}{4.5em}}}{\end{list}}

\newcounter{exercisenum}
\newcommand{\theexercise}{Exercise \theproblemsetnum-\arabic{exercisenum}}
\newenvironment{exercises}{
        \begin{list}{{\bf \theexercise. \hspace*{0.5em}}}
        {\setlength{\leftmargin}{0em}
         \setlength{\rightmargin}{0em}
         \setlength{\labelwidth}{0em}
         \setlength{\labelsep}{0em}
        \usecounter{exercisenum}}}{\end{list}}
\makeatletter
\newcommand{\exercise}{\item \let\@currentlabel=\theexercise}
\makeatother

\newcounter{exercisepartnum}[exercisenum]
%\newcommand{\problem}[1]{\medskip\mbox{}\newline\noindent{\bf Problem #1.}\hspace*{1em}}
%\newcommand{\exercise}[1]{\medskip\mbox{}\newline\noindent{\bf Exercise #1.}\hspace*{1em}}

\newenvironment{exerciseparts}{
        \begin{list}{{\bf (\alph{exercisepartnum})}}
        {\setlength{\leftmargin}{2.5em}
         \setlength{\rightmargin}{2.5em}
         \setlength{\labelsep}{0.5em}}}{\end{list}}
\newcommand{\exercisepart}{\addtocounter{exercisepartnum}{1}\item}


% Macros to make captions print with small type and 'Figure xx' in bold.
\makeatletter
\def\fnum@figure{{\bf Figure \thefigure}}
\def\fnum@table{{\bf Table \thetable}}
\let\@mycaption\caption
%\long\def\@mycaption#1[#2]#3{\addcontentsline{\csname
%  ext@#1\endcsname}{#1}{\protect\numberline{\csname 
%  the#1\endcsname}{\ignorespaces #2}}\par
%  \begingroup
%    \@parboxrestore
%    \small
%    \@makecaption{\csname fnum@#1\endcsname}{\ignorespaces #3}\par
%  \endgroup}
%\def\mycaption{\refstepcounter\@captype \@dblarg{\@mycaption\@captype}}
%\makeatother
\let\mycaption\caption
%\newcommand{\figcaption}[1]{\mycaption[]{#1}}

\newcounter{totalcaptions}
\newcounter{totalart}

\newcommand{\figcaption}[1]{\addtocounter{totalcaptions}{1}\caption[]{#1}}

% \psfigures determines what to do for figures:
%       0 means just leave vertical space
%       1 means put a vertical rule and the figure name
%       2 means insert the PostScript version of the figure
%       3 means put the figure name flush left or right
\newcommand{\psfigures}{0}
\newcommand{\spacefigures}{\renewcommand{\psfigures}{0}}
\newcommand{\rulefigures}{\renewcommand{\psfigures}{1}}
\newcommand{\macfigures}{\renewcommand{\psfigures}{2}}
\newcommand{\namefigures}{\renewcommand{\psfigures}{3}}

\newcommand{\figpart}[1]{{\bf (#1)}\nolinebreak[2]\relax}
\newcommand{\figparts}[2]{{\bf (#1)--(#2)}\nolinebreak[2]\relax}


\macfigures     % STATE

% When calling \figspace, make sure to leave a blank line afterward!!
% \widefigspace is for figures that are more than 28pc wide.
\newlength{\halffigspace} \newlength{\wholefigspace}
\newlength{\figruleheight} \newlength{\figgap}
\newcommand{\setfiglengths}{\ifnum\psfigures=1\setlength{\figruleheight}{\hruleheight}\setlength{\figgap}{1em}\else\setlength{\figruleheight}{0pt}\setlength{\figgap}{0em}\fi}
\newcommand{\figspace}[2]{\ifnum\psfigures=0\leavefigspace{#1}\else%
\setfiglengths%
\setlength{\wholefigspace}{#1}\setlength{\halffigspace}{.5\wholefigspace}%
\rule[-\halffigspace]{\figruleheight}{\wholefigspace}\hspace{\figgap}#2\fi}
\newlength{\widefigspacewidth}
% Make \widefigspace put the figure flush right on the text page.
\newcommand{\widefigspace}[2]{
\ifnum\psfigures=0\leavefigspace{#1}\else%
\setfiglengths%
\setlength{\widefigspacewidth}{28pc}%
\addtolength{\widefigspacewidth}{-\figruleheight}%
\setlength{\wholefigspace}{#1}\setlength{\halffigspace}{.5\wholefigspace}%
\makebox[\widefigspacewidth][r]{#2\hspace{\figgap}}\rule[-\halffigspace]{\figruleheight}{\wholefigspace}\fi}
\newcommand{\leavefigspace}[1]{\setlength{\wholefigspace}{#1}\setlength{\halffigspace}{.5\wholefigspace}\rule[-\halffigspace]{0em}{\wholefigspace}}

% Commands for including figures with macpsfig.
% To use these commands, documentstyle ``macpsfig'' must be specified.
\newlength{\macfigfill}
\makeatother
\newlength{\bbx}
\newlength{\bby}
\newcommand{\macfigure}[5]{\addtocounter{totalart}{1}
\ifnum\psfigures=2%
\setlength{\bbx}{#2}\addtolength{\bbx}{#4}%
\setlength{\bby}{#3}\addtolength{\bby}{#5}%
\begin{flushleft}
\ifdim#4>28pc\setlength{\macfigfill}{#4}\addtolength{\macfigfill}{-28pc}\hspace*{-\macfigfill}\fi%
\mbox{\psfig{figure=./#1.ps,%
bbllx=#2,bblly=#3,bburx=\bbx,bbury=\bby}}
\end{flushleft}%
\else\ifdim#4>28pc\widefigspace{#5}{#1}\else\figspace{#5}{#1}\fi\fi}
\makeatletter

\newlength{\savearraycolsep}
\newcommand{\narrowarray}[1]{\setlength{\savearraycolsep}{\arraycolsep}\setlength{\arraycolsep}{#1\arraycolsep}}
\newcommand{\normalarray}{\setlength{\arraycolsep}{\savearraycolsep}}

\newcommand{\hint}{{\em Hint:\ }}

% Macros from /th/u/clr/mac.tex

\newcommand{\set}[1]{\left\{ #1 \right\}}
\newcommand{\abs}[1]{\left| #1\right|}
\newcommand{\card}[1]{\left| #1\right|}
\newcommand{\floor}[1]{\left\lfloor #1 \right\rfloor}
\newcommand{\ceil}[1]{\left\lceil #1 \right\rceil}
\newcommand{\ang}[1]{\ifmmode{\left\langle #1 \right\rangle}
   \else{$\left\langle${#1}$\right\rangle$}\fi}
        % the \if allows use outside mathmode,
        % but will swallow following space there!
\newcommand{\paren}[1]{\left( #1 \right)}
\newcommand{\bracket}[1]{\left[ #1 \right]}
\newcommand{\prob}[1]{\Pr\left\{ #1 \right\}}
\newcommand{\Var}{\mathop{\rm Var}\nolimits}
\newcommand{\expect}[1]{{\rm E}\left[ #1 \right]}
\newcommand{\expectsq}[1]{{\rm E}^2\left[ #1 \right]}
\newcommand{\variance}[1]{{\rm Var}\left[ #1 \right]}
\renewcommand{\choose}[2]{{{#1}\atopwithdelims(){#2}}}
\def\pmod#1{\allowbreak\mkern12mu({\rm mod}\,\,#1)}
\newcommand{\matx}[2]{\left(\begin{array}{*{#1}{c}}#2\end{array}\right)}
\newcommand{\Adj}{\mathop{\rm Adj}\nolimits}

\newtheorem{theorem}{Theorem}
\newtheorem{lemma}[theorem]{Lemma}
\newtheorem{corollary}[theorem]{Corollary}
\newtheorem{xample}{Example}
\newtheorem{definition}{Definition}
\newenvironment{example}{\begin{xample}\rm}{\end{xample}}
\newcommand{\proof}{\noindent{\em Proof.}\hspace{1em}}
\def\squarebox#1{\hbox to #1{\hfill\vbox to #1{\vfill}}}
\newcommand{\qedbox}{\vbox{\hrule\hbox{\vrule\squarebox{.667em}\vrule}\hrule}}
\newcommand{\qed}{\nopagebreak\mbox{}\hfill\qedbox\smallskip}
\newcommand{\eqnref}[1]{(\protect\ref{#1})}

%%\newcommand{\twodots}{\mathinner{\ldotp\ldotp}}
\newcommand{\transpose}{^{\mbox{\scriptsize \sf T}}}
\newcommand{\amortized}[1]{\widehat{#1}}

\newcommand{\punt}[1]{}

%%% command for putting definitions into boldface
% New style for defined terms, as of 2/23/88, redefined by THC.
\newcommand{\defn}[1]{{\boldmath\textit{\textbf{#1}}}}
\newcommand{\defi}[1]{{\textit{\textbf{#1\/}}}}

\newcommand{\red}{\leq_{\rm P}}
\newcommand{\lang}[1]{%
\ifmmode\mathord{\mathcode`-="702D\rm#1\mathcode`\-="2200}\else{\rm#1}\fi}

%\newcommand{\ckt}[1]{\ifmmode\mathord{\mathcode`-="702D\sc #1\mathcode`\-="2200}\else$\mathord{\mathcode`-="702D\sc #1\mathcode`\-="2200}$\fi}
\newcommand{\ckt}[1]{\ifmmode \sc #1\else$\sc #1$\fi}

%% Margin notes - use \notesfalse to turn off notes.
\setlength{\marginparwidth}{0.6in}
\reversemarginpar
\newif\ifnotes
\notestrue
\newcommand{\longnote}[1]{
  \ifnotes
    {\medskip\noindent Note: \marginpar[\hfill$\Longrightarrow$]
      {$\Longleftarrow$}{#1}\medskip}
  \fi}
\newcommand{\note}[1]{
  \ifnotes
    {\marginpar{\tiny \raggedright{#1}}}
  \fi}


\newcommand{\reals}{\mathbbm{R}}
\newcommand{\integers}{\mathbbm{Z}}
\newcommand{\naturals}{\mathbbm{N}}
\newcommand{\rationals}{\mathbbm{Q}}
\newcommand{\complex}{\mathbbm{C}}

\newcommand{\oldreals}{{\bf R}}
\newcommand{\oldintegers}{{\bf Z}}
\newcommand{\oldnaturals}{{\bf N}}
\newcommand{\oldrationals}{{\bf Q}}
\newcommand{\oldcomplex}{{\bf C}}

\newcommand{\w}{\omega}                 %% for fft chapter

\newenvironment{closeitemize}{\begin{list}
{$\bullet$}
{\setlength{\itemsep}{-0.2\baselineskip}
\setlength{\topsep}{0.2\baselineskip}
\setlength{\parskip}{0pt}}}
{\end{list}}

% These are necessary within a {problems} environment in order to restore
% the default separation between bullets and items.
\newenvironment{normalitemize}{\setlength{\labelsep}{0.5em}\begin{itemize}}
                              {\end{itemize}}
\newenvironment{normalenumerate}{\setlength{\labelsep}{0.5em}\begin{enumerate}}
                                {\end{enumerate}}

%\def\eqref#1{Equation~(\ref{eq:#1})}
%\newcommand{\eqref}[1]{Equation (\ref{eq:#1})}
\newcommand{\eqreftwo}[2]{Equations (\ref{eq:#1}) and~(\ref{eq:#2})}
\newcommand{\ineqref}[1]{Inequality~(\ref{ineq:#1})}
\newcommand{\ineqreftwo}[2]{Inequalities (\ref{ineq:#1}) and~(\ref{ineq:#2})}

\newcommand{\figref}[1]{Figure~\ref{fig:#1}}
\newcommand{\figreftwo}[2]{Figures \ref{fig:#1} and~\ref{fig:#2}}

\newcommand{\liref}[1]{line~\ref{li:#1}}
\newcommand{\Liref}[1]{Line~\ref{li:#1}}
\newcommand{\lirefs}[2]{lines \ref{li:#1}--\ref{li:#2}}
\newcommand{\Lirefs}[2]{Lines \ref{li:#1}--\ref{li:#2}}
\newcommand{\lireftwo}[2]{lines \ref{li:#1} and~\ref{li:#2}}
\newcommand{\lirefthree}[3]{lines \ref{li:#1}, \ref{li:#2}, and~\ref{li:#3}}

\newcommand{\lemlabel}[1]{\label{lem:#1}}
\newcommand{\lemref}[1]{Lemma~\ref{lem:#1}} 

\newcommand{\exref}[1]{Exercise~\ref{ex:#1}}

\newcommand{\handref}[1]{Handout~\ref{#1}}

\newcommand{\defref}[1]{Definition~\ref{def:#1}}

% (1997.8.16: Victor Luchangco)
% Modified \hlabel to only get date and to use handouts counter for number.
%   New \handout and \handoutwithouttitle commands in newmac.tex use this.
%   The date is referenced by <label>-date.
%   (Retained old definition as \hlabelold.)
%   Defined \hforcelabel to use an argument instead of the handouts counter.

\newcounter{handouts}
\setcounter{handouts}{0}

\newcommand{\hlabel}[2]{%
\stepcounter{handouts}
{\edef\next{\write\@auxout{\string\newlabel{#1}{{\arabic{handouts}}{0}}}}\next}
\write\@auxout{\string\newlabel{#1-date}{{#2}{0}}}
}

\newcommand{\hforcelabel}[3]{%          Does not step handouts counter.
\write\@auxout{\string\newlabel{#1}{{#2}{0}}}
\write\@auxout{\string\newlabel{#1-date}{{#3}{0}}}}


% less ugly underscore
% --juang, 2008 oct 05
\renewcommand{\_}{\vrule height 0 pt depth 0.4 pt width 0.5 em \,}

\renewcommand{\name}{YOUR NAME HERE}

\usepackage{amsmath}
\usepackage{url}
\usepackage{graphicx}
\usepackage[all]{xy}
\usepackage[ruled,lined,noend,linesnumbered]{algorithm2e}
\SetKwIF{If}{ElseIf}{Else}{if}{}{else if}{else}{}%

\newcommand{\answer}{
 \par\medskip
 \textbf{Answer:}
}

\newcommand{\collaborators}{ \textbf{Collaborators:}
%%% COLLABORATORS START %%%
None.
%%% COLLABORATORS END %%%
}

\newcommand{\answerIa}{ \answer
%%% PROBLEM 1(a) ANSWER START %%%
0
%%% PROBLEM 1(a) ANSWER END %%%
}

\newcommand{\answerIb}{ \answer
%%% PROBLEM 1(b) ANSWER START %%%
0
%%% PROBLEM 1(b) ANSWER END %%%
}

\newcommand{\answerIc}{ \answer
%%% PROBLEM 1(c) ANSWER START %%%
0
%%% PROBLEM 1(c) ANSWER END %%%
}

\newcommand{\answerId}{ \answer
%%% PROBLEM 1(d) ANSWER START %%%
0
%%% PROBLEM 1(d) ANSWER END %%%
}
\newcommand{\answerIe}{ \answer
%%% PROBLEM 1(e) ANSWER START %%%
0
%%% PROBLEM 1(e) ANSWER END %%%
}
\newcommand{\answerIf}{ \answer
%%% PROBLEM 1(f) ANSWER START %%%
0
%%% PROBLEM 1(f) ANSWER END %%%
}
\newcommand{\answerIg}{ \answer
%%% PROBLEM 1(g) ANSWER START %%%
0
%%% PROBLEM 1(g) ANSWER END %%%
}
\newcommand{\answerIh}{ \answer
%%% PROBLEM 1(h) ANSWER START %%%
0
%%% PROBLEM 1(h) ANSWER END %%%
}
\newcommand{\answerIi}{ \answer
%%% PROBLEM 1(i) ANSWER START %%%
0
%%% PROBLEM 1(i) ANSWER END %%%
}
\newcommand{\answerIj}{ \answer
%%% PROBLEM 1(j) ANSWER START %%%
0
%%% PROBLEM 1(j) ANSWER END %%%
}
\newcommand{\answerIk}{ \answer
%%% PROBLEM 1(k) ANSWER START %%%
0
%%% PROBLEM 1(k) ANSWER END %%%
}
\newcommand{\answerIl}{ \answer
%%% PROBLEM 1(l) ANSWER START %%%
0
%%% PROBLEM 1(l) ANSWER END %%%
}
\newcommand{\answerIm}{ \answer
%%% PROBLEM 1(m) ANSWER START %%%
0
%%% PROBLEM 1(m) ANSWER END %%%
}
\newcommand{\answerIn}{ \answer
%%% PROBLEM 1(n) ANSWER START %%%
0
%%% PROBLEM 1(n) ANSWER END %%%
}
\newcommand{\answerIo}{ \answer
%%% PROBLEM 1(o) ANSWER START %%%
0
%%% PROBLEM 1(o) ANSWER END %%%
}
\newcommand{\answerIp}{ \answer
%%% PROBLEM 1(p) ANSWER START %%%
0
%%% PROBLEM 1(p) ANSWER END %%%
}
\newcommand{\answerIq}{ \answer
%%% PROBLEM 1(q) ANSWER START %%%
0
%%% PROBLEM 1(q) ANSWER END %%%
}
\newcommand{\answerIr}{ \answer
%%% PROBLEM 1(r) ANSWER START %%%
0
%%% PROBLEM 1(r) ANSWER END %%%
}
\newcommand{\answerIs}{ \answer
%%% PROBLEM 1(s) ANSWER START %%%
0
%%% PROBLEM 1(s) ANSWER END %%%
}
\newcommand{\answerIt}{ \answer
%%% PROBLEM 1(t) ANSWER START %%%
0
%%% PROBLEM 1(t) ANSWER END %%%
}
\newcommand{\answerIu}{ \answer
%%% PROBLEM 1(u) ANSWER START %%%
Your proof here.
%%% PROBLEM 1(u) ANSWER END %%%
}
\newcommand{\answerIIa}{ \answer 
%%% PROBLEM 2(a) ANSWER START %%%
0
%%% PROBLEM 2(a) ANSWER END %%%
}
\newcommand{\answerIIb}{ \answer
%%% PROBLEM 2(b) ANSWER START %%%
0
%%% PROBLEM 2(b) ANSWER END %%%
}
\newcommand{\answerIIc}{ \answer
%%% PROBLEM 2(c) ANSWER START %%%
0
%%% PROBLEM 2(c) ANSWER END %%%
}
\newcommand{\answerIId}{ \answer
%%% PROBLEM 2(d) ANSWER START %%%
0
%%% PROBLEM 2(d) ANSWER END %%%
}
\newcommand{\answerIIe}{ \answer
%%% PROBLEM 2(e) ANSWER START %%%
0
%%% PROBLEM 2(e) ANSWER END %%%
}
\newcommand{\answerIIf}{ \answer
%%% PROBLEM 2(f) ANSWER START %%%
0
%%% PROBLEM 2(f) ANSWER END %%%
}
\newcommand{\answerIIg}{ \answer
%%% PROBLEM 2(g) ANSWER START %%%
0
%%% PROBLEM 2(g) ANSWER END %%%
}
\newcommand{\answerIIh}{ \answer
%%% PROBLEM 2(h) ANSWER START %%%
0
%%% PROBLEM 2(h) ANSWER END %%%
}
\newcommand{\answerIIi}{ \answer
%%% PROBLEM 2(i) ANSWER START %%%
0
%%% PROBLEM 2(i) ANSWER END %%%
}
\newcommand{\answerIIj}{ \answer
%%% PROBLEM 2(j) ANSWER START %%%
method\_name
%%% PROBLEM 2(j) ANSWER END %%%
}
\newcommand{\answerIIk}{ \answer
%%% PROBLEM 2(k) ANSWER START %%%
0
%%% PROBLEM 2(k) ANSWER END %%%
}

\setlength{\oddsidemargin}{0pt}
\setlength{\evensidemargin}{0pt}
\setlength{\textwidth}{6.5in}
\setlength{\topmargin}{0in}
\setlength{\textheight}{8.5in}

\newcommand{\theproblemsetnum}{3-2}
\newcommand{\releasedate}{September 17, 2020}
\newcommand{\partaduedate}{September 28}
\newcommand{\critiqueduedate}{October 5}
\newcommand{\tabUnit}{3ex}
\newcommand{\tabT}{\hspace*{\tabUnit}}

\begin{document}

\handout{Problem Set \theproblemsetnum}{\releasedate}

\textbf{Both theory and programming questions} are due {\bf \partaduedate} at
{\bf 11:59PM}.
%
Please submit your solutions on Canvas, as two separate files: one image containing
your write-up, and one zip containing your programming solutions.

We will provide the solutions to the problem set after the problem set
is due, which you will use to find any errors in the proof that you submitted.
You will need to submit a critique of your solutions by \critiqueduedate.
Your grade will be based on both your solutions and
your critique of the solutions.

\medskip

\hrulefill

\collaborators

\begin{problems}

\problem \points{45} \textbf{Range Queries}

Microware is preparing to launch a new database product, NoSQL Server, aimed at the
real-time analytics market. Web application analytics record information (e.g., the
times when users visit the site, or how much does it take the server to produce a
HTTP response), and can display the information using a Pretty Graph$^{\texttt{TM}}$,
so that CTOs can claim that they’re using data to back their decisions.

NoSQL Server databases will support a special kind of index, called a range index,
to speed up the operations needed to build a Pretty Graph$^{\texttt{TM}}$ out of data.
Microware has interviewed you during the fall Career Fair, and immediately hired you
as a consultant and asked you to help the NoSQL Server team design the range index.

The range index must support fast (sub-linear) insertions, to keep up with Web
application traffic. The first step in the Pretty Graph$^{\texttt{TM}}$ algorithm is
finding the minimum and maximum values to be plotted, to set up the graph’s
horizontal axis. So the range index must also be able to compute the minimum and
maximum over all keys quickly (in sub-linear time).

\begin{problemparts}
  \problempart \points{1} Given the constraints above, what data structure covered
  in lectures so far should be used for the range index? Microware engineers need
  to implement range indexes, so choose the simplest data struture that meets the
  requirements.
    \begin{enumerate}
      \item Min-Heap
      \item Max-Heap
      \item Binary Search Tree (BST)
      \item AVL Trees
      \item B-Trees
    \end{enumerate}
\answerIa

  \problempart \points{1} How much time will it take to insert a key in the range
  index?
    \begin{enumerate}
      \item $O(1)$
      \item $O(\log(\log N))$
      \item $O(\log N)$
      \item $O(\log^2 N)$
      \item $O(\sqrt{N})$
    \end{enumerate}
\answerIb
    
  \problempart \points{1} How much time will it take to find the minimum key in
  the range index?
    \begin{enumerate}
      \item $O(1)$
      \item $O(\log(\log N))$
      \item $O(\log N)$
      \item $O(\log^2 N)$
      \item $O(\sqrt{N})$
    \end{enumerate}
\answerIc 

  \problempart \points{1} How much time will it take to find the maximum key in
  the range index?
    \begin{enumerate}
      \item $O(1)$
      \item $O(\log(\log N))$
      \item $O(\log N)$
      \item $O(\log^2 N)$
      \item $O(\sqrt{N})$
    \end{enumerate}
\answerId

\end{problemparts}

The main work of the Pretty Graph$^{\texttt{TM}}$ algorithm is drawing the bars in the
graph. A bar shows how many data points there are between two values. For example,
in order to produce the visitor graph that is the hallmark of Google Analytics, the
range index would record each time that someone uses the site, and a bar would count
the visiting times between the beginning and the ending of a day. Therefore, the
range index needs to support a fast (sub-linear time) $\texttt{COUNT}(l, h)$ query
that returns the number of keys in the index that are between $l$ and $h$
(formally, keys $k$ such that $l \leq k \leq h$). Note that keys are unique; if $l$
exists in the range index, it appears exactly once.

Your instinct tells you that $\texttt{COUNT}(l, h)$ can be easily implemented on top
of a simpler query, $\texttt{RANK}(x)$, which returns the number of keys in
the index that are smaller or equal to $x$ (informally, if the keys were listed in
ascending order, $x$'s rank would indicate its position in the sorted array).

\begin{problemparts}

  \problempart \points{1} Assuming $l < h$, and both $l$ and $h$ exist in the
  index, $\texttt{COUNT}(l, h)$ is
    \begin{enumerate}
      \item $\texttt{RANK}(l)-\texttt{RANK}(h)-1$
      \item $\texttt{RANK}(l)-\texttt{RANK}(h)$
      \item $\texttt{RANK}(l)-\texttt{RANK}(h)+1$
      \item $\texttt{RANK}(h)-\texttt{RANK}(l)-1$
      \item $\texttt{RANK}(h)-\texttt{RANK}(l)$
      \item $\texttt{RANK}(h)-\texttt{RANK}(l)+1$
      \item $\texttt{RANK}(h)+\texttt{RANK}(l)-1$
      \item $\texttt{RANK}(h)+\texttt{RANK}(l)$
      \item $\texttt{RANK}(h)+\texttt{RANK}(l)+1$
    \end{enumerate}
\answerIe

  \problempart \points{1} Assuming $l < h$, and $h$ exists in the index, but $l$ does
  not exist in the index, $\texttt{COUNT}(l, h)$ is
  \begin{enumerate}
      \item $\texttt{RANK}(l)-\texttt{RANK}(h)-1$
      \item $\texttt{RANK}(l)-\texttt{RANK}(h)$
      \item $\texttt{RANK}(l)-\texttt{RANK}(h)+1$
      \item $\texttt{RANK}(h)-\texttt{RANK}(l)-1$
      \item $\texttt{RANK}(h)-\texttt{RANK}(l)$
      \item $\texttt{RANK}(h)-\texttt{RANK}(l)+1$
      \item $\texttt{RANK}(h)+\texttt{RANK}(l)-1$
      \item $\texttt{RANK}(h)+\texttt{RANK}(l)$
      \item $\texttt{RANK}(h)+\texttt{RANK}(l)+1$
    \end{enumerate}
\answerIf

  \problempart \points{1} Assuming $l < h$, and $l$ exists in the index but, $h$ does
  not exist in the index, $\texttt{COUNT}(l, h)$ is
    \begin{enumerate}
      \item $\texttt{RANK}(l)-\texttt{RANK}(h)-1$
      \item $\texttt{RANK}(l)-\texttt{RANK}(h)$
      \item $\texttt{RANK}(l)-\texttt{RANK}(h)+1$
      \item $\texttt{RANK}(h)-\texttt{RANK}(l)-1$
      \item $\texttt{RANK}(h)-\texttt{RANK}(l)$
      \item $\texttt{RANK}(h)-\texttt{RANK}(l)+1$
      \item $\texttt{RANK}(h)+\texttt{RANK}(l)-1$
      \item $\texttt{RANK}(h)+\texttt{RANK}(l)$
      \item $\texttt{RANK}(h)+\texttt{RANK}(l)+1$
    \end{enumerate}
\answerIg

  \problempart \points{1} Assuming $l < h$, and neither $l$ nor $h$ exist in the
  index, $\texttt{COUNT}(l, h)$ is
    \begin{enumerate}
      \item $\texttt{RANK}(l)-\texttt{RANK}(h)-1$
      \item $\texttt{RANK}(l)-\texttt{RANK}(h)$
      \item $\texttt{RANK}(l)-\texttt{RANK}(h)+1$
      \item $\texttt{RANK}(h)-\texttt{RANK}(l)-1$
      \item $\texttt{RANK}(h)-\texttt{RANK}(l)$
      \item $\texttt{RANK}(h)-\texttt{RANK}(l)+1$
      \item $\texttt{RANK}(h)+\texttt{RANK}(l)-1$
      \item $\texttt{RANK}(h)+\texttt{RANK}(l)$
      \item $\texttt{RANK}(h)+\texttt{RANK}(l)+1$
    \end{enumerate}
\answerIh

\end{problemparts}

Now that you know how to reduce a \texttt{COUNT()} query to a constant number of
\texttt{RANK()} queries, you want to figure out how to implement \texttt{RANK()} in
sub-linear time. None of the tree data structures that you studied in class supports
optimized \texttt{RANK()} out of the box, but you just remembered that tree data
structures can respond to some queries faster if the nodes are cleverly augmented
with some information.

\begin{problemparts}

  \problempart \points{1} In order to respond to \texttt{RANK()} queries in 
  sub-linear time, each node node in the tree will be augmented with an extra field,
  $node.\gamma$. Keep in mind that for a good augmentation, the extra information for
  a node should be computed in $O(1)$ time, based on other properties of the node,
  and on the extra information stored in the node’s subtree. The meaning of 
  $node.\gamma$ is
    \begin{enumerate}
      \item the minimum key in the subtree rooted at $node$
      \item the maximum key in the subtree rooted at $node$
      \item the height of the subtree rooted at $node$
      \item the number of nodes in the subtree rooted at $node$
      \item the rank of $node$
      \item the sum of the keys in the subtree rooted at $node$
    \end{enumerate}
\answerIi

  \problempart \points{1} How many extra $bits$ of storage per node does the
  augmentation above require?
    \begin{enumerate}
      \item $O(1)$
      \item $O(\log(\log N))$
      \item $O(\log N)$
      \item $O(\log^2 N)$
      \item $O(\sqrt{N})$
    \end{enumerate}
\answerIj

\end{problemparts}

The following questions refer to the tree below.

\xymatrix @=1em{
  &&&&&&& *++[o][F-]{N_1} \ar[lllld]\ar[rrrrd] &&&&&&& \\
  &&& *++[o][F-]{N_2} \ar[lld]\ar[rrd] &&&&&&&& 
    *++[o][F-]{N_8} \ar[lld]\ar[rrd] &&& \\
  & *++[o][F-]{N_3} \ar[ld]\ar[rd] &&&& *++[o][F-]{N_6} \ar[ld] &&&&
    *++[o][F-]{N_9} &&&& *++[o][F-]{N_{10}} & \\
  *++[o][F-]{N_4} && *++[o][F-]{N_5} && *++[o][F-]{N_7} && && && && && \\
}

\begin{problemparts}

  \problempart \points{1} $N_4.\gamma$ is
    \begin{enumerate}
      \item $0$
      \item $1$
      \item $2$
      \item the key at $N_4$
    \end{enumerate}
\answerIk

  \problempart \points{1} $N_3.\gamma$ is
    \begin{enumerate}
      \item $1$
      \item $2$
      \item $3$
      \item the key at $N_4$
      \item the key at $N_5$
      \item the sum of the keys at $N_3\dots N_5$
    \end{enumerate}
\answerIl

  \problempart \points{1} $N_2.\gamma$ is
    \begin{enumerate}
      \item $2$
      \item $3$
      \item $4$
      \item $6$
      \item the key at $N_4$
      \item the key at $N_7$
      \item they sum of the keys at $N_3\dots N_7$
    \end{enumerate}
\answerIm

  \problempart \points{1} $N_1.\gamma$ is
    \begin{enumerate}
      \item $3$
      \item $6$
      \item $7$
      \item $10$
      \item the key at $N_4$
      \item the key at $N_{10}$
      \item the sum of the keys at $N_1\dots N_{10}$
    \end{enumerate}
\answerIn

  \problempart \points{6} Which fo the following functions need to be modified
  to update $\gamma$? List all that apply. If the function does not apply to
  the data structure used to implement the range index, it does not need to be
  modified.
    \begin{enumerate}
      \item \texttt{INSERT}
      \item \texttt{DELETE}
      \item \texttt{ROTATE-LEFT}
      \item \texttt{ROTATE-RIGHT}
      \item \texttt{REBALANCE}
      \item \texttt{HEAPIFY}
    \end{enumerate}
\answerIo

  \problempart \points{1} What is the running time of a \texttt{COUNT()}
  implementation based on \texttt{RANK()} with the augmentations made to the
  tree?
    \begin{enumerate}
      \item $O(1)$
      \item $O(\log(\log N))$
      \item $O(\log N)$
      \item $O(\log^2 N)$
      \item $O(\sqrt{N})$
    \end{enumerate}
\answerIp

\end{problemparts}

After the analytics data is plotted using Pretty Graph$^{\texttt{TM}}$, the CEO can
hover the mouse cursor over one of the bars, and the graph will show a tooltip with
the information represented by that bar. To support this operation, the range index
needs to support a $\texttt{LIST}(l, h)$ operation that returns all the keys between
l and h as quickly as possible.

$\texttt{LIST}(l, h)$ cannot be sub-linear in the worst case, because 
$\texttt{LIST}(-\infty, +\infty)$ must return all the keys in the index, which takes
$\Omega(n)$ time. However, if \texttt{LIST} only has to return a few elements, we
would like it to run in sub-linear time. We formalize this by stating that
\texttt{LIST}'s running time should be $T(N) + \Theta(L)$, where $L$ is the length
of the list of keys output by \texttt{LIST}, and $T(N)$ is sub-linear.

Inspiration strikes again, and you find yourself with the following pseudocode
for \texttt{LIST}.

\begin{algorithm}[H]
\caption{$\texttt{LIST}(tree, l, h)$}
  $lca = \texttt{LCA}(tree, l, h)$ \\
  $result = []$ \\
  $\texttt{NODE-LIST}(lca, l, h, result)$ \\
  \Return{$result$}
\end{algorithm}

\begin{algorithm}[H]
\caption{$\texttt{NODE-LIST}(node, l, h, result)$}
  \If{$node == \texttt{NIL}$}
  {
    \Return
  }
  \If{$l\leq node.key \text{ and } node.key \leq h$}
  {
    $\texttt{ADD-KEY}(result, node.key)$
  }
  \If{$node.key \geq l$}
  {
    $\texttt{NODE-LIST}(node.left, l, h, result)$
  }
  \If{$node.key \leq h$}
  {
    $\texttt{NODE-LIST}(node.right, l, h, result)$
  }
\end{algorithm}

\begin{algorithm}[H]
\caption{$\texttt{LCA}(tree, l, h)$}
  $node = tree.root$ \\
  \While {
    $node \neq \texttt{NIL} \textbf{ and }
    \big(l > node.key \textbf{ or } h < node.key\big)$
  }
  {
    \If{$l < node.key$}
    {
      $node = node.left$
    }
    \Else{
      $node = node.right$
    }
  }
  \Return $node$
\end{algorithm}

\begin{problemparts}

  \problempart \points{1} \texttt{LCA} most likely stands for
    \begin{enumerate}
      \item last common ancestor
      \item lowest common ancestor
      \item low cost airline
      \item life cycle assessment
      \item logic cell array
    \end{enumerate}
\answerIq

  \problempart \points{1} The running time of $\texttt{LCA}(l,h)$ for trees
  used by the range index is
    \begin{enumerate}
      \item $O(1)$
      \item $O(\log(\log N))$
      \item $O(\log N)$
      \item $O(\log^2 N)$
      \item $O(\sqrt{N})$
    \end{enumerate}
\answerIr

  \problempart \points{1} Assuming that $\texttt{ADD-KEY}$ runs in $O(1)$ time,
  and that $\texttt{LIST}$ returns a list of $L$ keys, the running time of
  $\texttt{NODE-LIST}$ call at line 3 of $\texttt{LIST}$ is
   \begin{enumerate}
      \item $O(1)$
      \item $O(\log(\log N))$
      \item $O(\log N)$
      \item $O(\log^2 N)$
      \item $O(\sqrt{N})$
      \item $O(1) + O(L)$
      \item $O(\log(\log N)) + O(L)$
      \item $O(\log N) + O(L)$
      \item $O(\log^2N) + O(L)$
      \item $O(\sqrt{N}) + O(L)$
    \end{enumerate}
\answerIs

  \problempart \points{1} Assuming that $\texttt{ADD-KEY}$ runs in $O(1)$ time,
  and that $\texttt{LIST}$ returns a list of $L$ keys, the running time of
  $\texttt{LIST}$ is
    \begin{enumerate}
      \item $O(1)$
      \item $O(\log(\log N))$
      \item $O(\log N)$
      \item $O(\log^2 N)$
      \item $O(\sqrt{N})$
      \item $O(1) + O(L)$
      \item $O(\log(\log N)) + O(L)$
      \item $O(\log N) + O(L)$
      \item $O(\log^2N) + O(L)$
      \item $O(\sqrt{N}) + O(L)$
    \end{enumerate}
\answerIt
    
  \problempart \points{20} Prove that $\texttt{LCA}$ is correct (i.e. that if
  $l$ and $h$ have common ancestors in the tree, it returns the lowest such
  ancestor, and otherwise it returns $\texttt{NIL}$).
\answerIu

\end{problemparts}


\problem \points{55} \textbf{Digital Circuit Layout}

Your AMDtel internship is off to a great start! The optimized circuit simulator
cemented your reputation as an algorithms whiz. Your manager capitalized on your
success, and promised to deliver the Bullfield chip a few months ahead of schedule.
Thanks to your simulator optimizations, the engineers have finished the logic-level
design, and are currently working on laying out the gates on the chip. Unfortunately,
the software that verifies the layout is taking too long to run on the preliminary
Bullfield layouts, and this is making the engineers slow and unhappy. Your manager is
confident in your abilities to speed it up, and promised that you’ll “do your magic”
again, in “one week, two weeks tops”.

A chip consists of logic gates, whose input and output terminals are connected by
wires (very thin conductive traces on the silicon substrate). AMDtel’s high-yield
manufacturing process only allows for horizontal or vertical wires. Wires must not
cross each other, so that the circuit will function according to its specification.
This constraint is checked by the software tool that you will optimize. The
topologies required by complex circuits are accomplished by having dozens of layers
of wires that do not touch each other, and the tool works on one layer at a time.

\begin{problemparts}

\problempart \points{1} Run the code under the python profiler with the command
below, and identify the method that takes up most of the CPU time. If two
methods have similar CPU usage times, ignore the simpler one.

\texttt{python -m cProfile -s time circuit2.py < tests/10grid\_s.in}

\textit{Warning:} the command above can take 15-60 minutes to complete, and
bring the CPU usage to 100\% on one of your cores. Plan accordingly.

What is the name of the method with the highest CPU usage?
\answerIIa

\problempart \points{1} How many times is the method called?
\answerIIb

\end{problemparts}

The method that has the performance bottleneck is called from the 
\texttt{CrossVerifier} class. Upon reading the class, it seems that the original
author was planning to implement a $sweep-line$ algorithm, but couldn’t figure out
the details, and bailed and implemented an inefficient method at the last minute.
Fortunately, most of the infrastructure for a fast sweep-line algorithm is still
in place. Furthermore, you notice that the source code contains a trace of the
working sweep-line algorithm, in the \texttt{good\_trace.jsonp} file.

Sweep-line algorithms are popular in computational geometry. Conceptually, such an
algorithm sweeps a vertical line left to right over the plane containing the input
data, and performs operations when the line “hits” points of interest in the input.
This is implemented by generating an array containing all the points of interest, and
then sorting them according to their position along the horizontal axis ($x$
coordinate).

Read the source for \texttt{CrossVerifier} to get a feel for how the sweep-line
infrastructure is supposed to work, and look at the good trace in the visualizer
that we have provided for you. To see the good trace, copy
\texttt{good\_trace.jsonp} to \texttt{trace.jsonp}:

\texttt{cp good\_trace.jsonp trace.jsonp}

On Windows, use this command instead (or use a bash terminal):

\texttt{copy good\_trace.jsonp trace.jsonp}

Then use Google Chrome to open \texttt{visualizer/bin/visualizer.html}.

The questions below refer to the fast sweep-line algorithm shown in
\texttt{good\_trace.jsonp}, not to the slow algorithm hacked together
in \texttt{circuit2.py}.

\begin{problemparts}

\problempart \points{5} The $x$ coordinates of points of interest in the input
are (mark each option as true or false):
\begin{enumerate}
  \item the $x$ coordinates of the left endpoints of horizontal wires
  \item the $x$ coordinates of the right endpoints of horizontal wires
  \item the $x$ coordinates of the midpoints of horizontal wires
  \item the $x$ coordinates where vertical wires cross horizontal wires
  \item the $x$ coordinates of vertical wires
\end{enumerate}
\answerIIc 

\problempart \points{1} When the sweep line hits the $x$ coordinate of the
left endpoint of a horizontal wire
\begin{enumerate}
  \item the wire is added to the range index
  \item the wire is removed from the range index
  \item a range index query is performed
  \item nothing happens
\end{enumerate}
\answerIId

\problempart \points{1} When the sweep line hits the $x$ coordinate of the right
endpoint of a horizontal wire
\begin{enumerate}
  \item the wire is added to the range index
  \item the wire is removed from the range index
  \item a range index query is performed
  \item nothing happens
\end{enumerate}
\answerIIe

\problempart \points{1} When the sweep line hits the $x$ coordinate of the midpoint
of a horizontal wire
\begin{enumerate}
  \item the wire is added to the range index
  \item the wire is removed from the range index
  \item a range index query is performed
  \item nothing happens
\end{enumerate}
\answerIIf

\problempart When the sweep line hits the $x$ coordinate of a vertical wire
\begin{enumerate}
  \item the wire is added to the range index
  \item the wire is removed from the range index
  \item a range index query is performed
  \item nothing happens
\end{enumerate}
\answerIIg

\problempart \points{1} What is a good invariant for the sweep-line algorithm?
\begin{enumerate}
  \item the range index holds all of the horizontal wires to the left of the
    sweep line
  \item the range index holds all of the horizontal wires "cut" by the sweep line
  \item the range index holds all of the horizontal wires to the right of the
    sweep line
  \item the range index holds all of the wires to the left of the sweep line
  \item the range index holds all of the wires to the right of the sweep line
\end{enumerate}
\answerIIh

\problempart \points{1} When a wire is added to the range index, what is its
corresponding key?
\begin{enumerate}
  \item the $x$ coordinate of the wire's midpoint
  \item the $y$ coordinate of the wire's midpoint
  \item the segment's length
  \item the $x$ coordinate of interest that will remove the wire from the index
\end{enumerate}

\end{problemparts}

Modify \texttt{CrossVerifier} in \texttt{circuit2.py} to implement the sweep-line
algorithm discussed above. If you maintain the current code structure, you’ll be able
to use our visualizer to debug your implementation. To use our visualizer, first
produce a trace:

\texttt{TRACE=jsonp python circuit2.py < tests/5logo.in > trace.jsonp}

On windows, run the following command instead to produce a trace:

\texttt{circuit2\_jsonp.bat < tests/5logo.in > trace.jsonp}

Then use Google Chrome to open \texttt{visualizer/bin/visualizer.html}.

Note that there are tests provided other than \texttt{tests/5logo.in}, whose trace
is produced above.

\begin{problemparts}

\problempart \points{1} Run your modified code under the python profiler again, using
the same test as before, and identify the method that takes up the most CPU time.

What is the name of the method with the highest CPU usage now? If two methods have
similar CPU usage times, ignore the simpler one.
\answerIIj

\problempart \points{1} How many times is the method called?
\answerIIk

\problempart \points{40} Modify \texttt{circuit2.py} to implement a data structure
that has better asymptotic running time for the operation above. Keep in mind that
the tool has two usage scenarios:

\begin{itemize}
  \item Every time an engineer submits a change to one of the Bullhorn wire layers,
    the tool must analyze the layer and report the number of wire crossings. In this
    late stage of the project, the version control system will automatically reject
    the engineer’s change if it causes the number of wire crossings to go up over the
    previous version.
  \item Engineers working on the wiring want to see the pairs of wires that intersect, so they know where to focus ther efforts. To activate this detailed output, generate a trace using the following command.\\
  \texttt{TRACE=list python circuit2.py < tests/6list\_logo.in}\\
  On Windows, run this command instead:\\
  \texttt{circuit2\_list.bat < tests/6list\_logo.in}\\
  Again, note that the chosen test \texttt{6list\_logo.in} can be changed. The
  important difference between these commands and those given earlier to generate
  traces is that \texttt{TRACE=jsonp} has been replaced with \texttt{TRACE=list}
  (or, for Windows, that \texttt{circuit2\_json.bat} has been replaced with
  \texttt{circuit2\_list.bat}).
  \item Correctness can be tested exactly as it was in the previous assignment
\end{itemize}

When your code passes all tests, and runs reasonably fast (the tests should
complete in less than 60 seconds on any reasonably recent computer), you are
done. The only code that you need to submit is your modified \texttt{circuit2.py}.

\end{problemparts}

\end{problems}
\end{document}
