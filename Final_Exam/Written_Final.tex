%
% 354 Final Exam (Written Portion)
%
\documentclass[12pt,twoside]{article}

\input{macros}
\renewcommand{\name}{YOUR NAME HERE}

\usepackage{amsmath}
\usepackage{amsfonts}
\usepackage[ruled,lined,noend,linesnumbered]{algorithm2e}
\SetKwIF{If}{ElseIf}{Else}{if}{}{else if}{else}{}
\SetKwFor{For}{for}{}{}
\SetKwInOut{IN}{\texttt{IN}}
\SetKwInOut{OUT}{\texttt{OUT}}
\NoCaptionOfAlgo

\newcommand{\answer}{
 \par\medskip
 \textbf{Answer}
}

\newcommand{\answerI}{ \answer
%%% PROBLEM 1 ANSWER START %%%

%%% PROBLEM 1 ANSWER END %%%
}

\newcommand{\answerII}{ \answer
%%% PROBLEM 2 ANSWER START %%%

%%% PROBLEM 2 ANSWER END %%%
}

\newcommand{\answerIII}{ \answer
%%% PROBLEM 3 ANSWER START %%%

%%% PROBLEM 3 ANSWER END %%%
}

\setlength{\oddsidemargin}{0pt}
\setlength{\evensidemargin}{0pt}
\setlength{\textwidth}{6.5in}
\setlength{\topmargin}{0in}
\setlength{\textheight}{8.5in}

\newcommand{\theproblemsetnum}{3}
\newcommand{\releasedate}{Dec 8, 2020}
\newcommand{\partaduedate}{Thursday, Dec 10}
\newcommand{\tabUnit}{3ex}
\newcommand{\tabT}{\hspace*{\tabUnit}}

\begin{document}

\handout{Final Exam}{\releasedate}

This final is due {\bf \partaduedate} at {\bf 11:59PM}. Note that you 
must take an interview exam in addition to this written exam; failure to
interview will result in a failing grade on the final exam, of which this
is only half.

\medskip

\hrulefill

\begin{problems}

\problem \points{15} \textbf{Proof of Correctness}

Consider a problem in which you have a list $J$ of jobs,
each of which takes the same amount of time to complete. You need to
schedule these jobs in array $S$, indexed from $0$ to $n$. Each
job occupies a single index in the array, because they each take the
same amount of time to complete. Each job $j \in J$ has:

\begin{itemize}
  \item {\bf deadline} $j.deadline$ from $0$ to $n$, denoting the
    latest index in $S$ at which it can be scheduled.
  \item {\bf profit} $j.profit$, denoting the profit or gain
    from completing job $j$.
\end{itemize}

The goal is to maximize the sum of profits of the jobs scheduled in
$S$. $\abs{J}$ can be greater than $\abs{S}$, so it might not be
possible to schedule all jobs. There is no penalty for leaving
a job undone.

Consider the following algorithm:

\begin{algorithm}[H]
  \caption{Job Scheduling}
  \IN{List $J$ of jobs, max index $n$}
  $S \leftarrow \text{an array indexed $0$ to $n$, with $null$ at each index}$\\
  Sort $J$ in non-increasing order of profits\\
  \For{$i$ from $0$ to $n$}{
    Find the largest $t$ such that $S[t] = null$ and $t\leq J[i].deadline$
      (if one exists) \\
    \If{an index $t$ was found}{
      $S[t] \leftarrow J[i]$
    }
  }
  \OUT{$S$ maximizes the profit of scheduled jobs}
\end{algorithm}

Note that $\leftarrow$ is used to denote assignment; $x\leftarrow y$ means that
$x$ is assigned the value of $y$.

Prove that the algorithm is correct.

HINT: A schedule $S$ is {\bf promising} if it can be extended to an optimal
schedule by scheduling jobs that have not yet been considered. If $S$ is
promising after all jobs have been considered, then $S$ is an optimal
schedule.

\answerI

\newpage

\problem \points{15} \textbf{Time Complexity}

Derive the time complexity of the following algorithm. Show your work.

\begin{algorithm}[H]
  \caption{MULT}
  \IN{$x$ and $y$, two $n$-bit binary numbers}
  \If{$b = 1$}{
    \If{$x = 1 \wedge y = 1$}{
      \Return $1$
    }
    \Else {
      \Return $0$
    }
  }
  $(x_1, x_0) \leftarrow (\text{first } \lceil n/2 \rceil \text{ bits }, 
    \text{ last } \lfloor n/2 \rfloor \text{ bits })$ of $x$ \\
  $(y_1, y_0) \leftarrow (\text{first } \lceil n/2 \rceil \text{ bits }, 
    \text{ last } \lfloor n/2 \rfloor \text{ bits })$ of $y$ \\
  $z_1 \leftarrow \textsc{MULT}(x_1 + x_0, y_1 + y_0)$ \\
  $z_2 \leftarrow \textsc{MULT}(x_1, y_1)$ \\
  $z_3 \leftarrow \textsc{MULT}(x_0, y_0)$ \\
  \Return $z_2 \cdot 2^n + (z_1 - z_2 - z_3) \cdot 2^{\lceil n/2 \rceil}
     + z_3$ \\
  \OUT{The product of $x$ and $y$}
\end{algorithm}

Some details:
\begin{itemize}
  \item All additions and subtractions take $O(n)$ time.
  \item In the final return line, the multiplications by $2^{\text{some power}}$
    are actually bit shifts, and take $O(n)$ time.
  \item The base case takes $O(1)$ time.
\end{itemize}

\answerII

\newpage

\problem \points{20} \textbf{Algorithmic Design}

Given a sequence $R = \{r_1, r_2, ..., r_n\}$ of real numbers, let $S_{ij}$ denote
the sum $r_i + r_{i+1} + ... + r_{j-1} + r_j$; in other words, $S_{ij}$ is the sum
of all elements in $R$ from index $i$ to index $j$.

Write an algorithm which takes a sequence of real numbers $R$ as an input and
efficienltly finds a maximum sum $S_{ij}$.

For instance, given the sequence $R = \{ 1, -2, 1, 2, 4, -9, 6 \}$, your algorithm
should output the number $7$, which results from adding $1$, $2$ and $4$; this is the
largest sum which can be acquired by adding consecutive elements of $R$.

Note that while this is easily doable on $O(n^2)$ time by simply finding the sum
of every consecutive subsequence, it is doable in $O(n)$ time if the subproblems
are chosen more carefully.

Your pseudocode should be concisely and clearly written; convoluted, messy, or
unnecessarily complex solutions will lose points.

\answerIII

\end{problems}
\end{document}
