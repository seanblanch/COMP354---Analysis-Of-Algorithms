%
% 354 Homework 4
%
\documentclass[12pt,twoside]{article}

% Cross-references for handout numbers.

% Updated to include SMA course for Fall 2001 -- cel

\newcommand{\name}{}


\usepackage{latexsym}
%\usepackage{bbm}
\usepackage{times,url}
\usepackage{clrscode3e}

\newcommand{\csucist}[1]{\begin{description}
\item[CSUCI students:] #1
\end{description}}

\newcommand{\profs}{Ryan McIntyre}
\newcommand{\subj}{354}

\newlength{\toppush}
\setlength{\toppush}{2\headheight}
\addtolength{\toppush}{\headsep}

\newcommand{\htitle}[2]{\noindent\vspace*{-\toppush}\newline\parbox{6.5in}
{\textit{Analysis of Algorithms: 354}\hfill\name\newline
CSU Channel Islands \hfill #2\newline
\profs\hfill #1 \vspace*{-.5ex}\newline
\mbox{}\hrulefill\mbox{}}\vspace*{1ex}\mbox{}\newline
\begin{center}{\Large\bf #1}\end{center}}

\newcommand{\handout}[2]{\thispagestyle{empty}
 \markboth{#1}{#1}
 \pagestyle{myheadings}\htitle{#1}{#2}}

\newcommand{\htitlewithouttitle}[2]{\noindent\vspace*{-\toppush}\newline\parbox{6.5in}
{\textit{Analysis of Algorithms}\hfill#2\newline
CSU Channel Islands \hfill 354\newline
%Singapore-MIT Alliance \hfill SMA5503\newline
\profs\hfill Handout #1\vspace*{-.5ex}\newline
\mbox{}\hrulefill\mbox{}}\vspace*{1ex}\mbox{}\newline}

\newcommand{\handoutwithouttitle}[2]{\thispagestyle{empty}
 \markboth{Handout \protect\ref{#1}}{Handout \protect\ref{#1}}
 \pagestyle{myheadings}\htitlewithouttitle{\protect\ref{#1}}{#2}}

\newcommand{\exam}[2]{% parameters: exam name, date
 \thispagestyle{empty}
 \markboth{\subj\ #1\hspace{1in}Name\hrulefill\ \ }%
          {\subj\ #1\hspace{1in}Name\hrulefill\ \ }
 \pagestyle{myheadings}\examtitle{#1}{#2}
 \renewcommand{\theproblem}{Problem \arabic{problemnum}}
}
\newcommand{\examsolutions}[3]{% parameters: handout, exam name, date
 \thispagestyle{empty}
 \markboth{Handout \protect\ref{#1}: #2}{Handout \protect\ref{#1}: #2}
% \pagestyle{myheadings}\htitle{\protect\ref{#1}}{#2}{#3}
 \pagestyle{myheadings}\examsolutionstitle{\protect\ref{#1}} {#2}{#3}
 \renewcommand{\theproblem}{Problem \arabic{problemnum}}
}
\newcommand{\examsolutionstitle}[3]{\noindent\vspace*{-\toppush}\newline\parbox{6.5in}
{\textit{Analysis of Algorithms}\hfill#3\newline
CSU Channel Islands \hfill 354\newline
%Singapore-MIT Alliance \hfill SMA5503\newline
\profs\hfill Handout #1\vspace*{-.5ex}\newline
\mbox{}\hrulefill\mbox{}}\vspace*{1ex}\mbox{}\newline
\begin{center}{\Large\bf #2}\end{center}}

\newcommand{\takehomeexam}[2]{% parameters: exam name, date
 \thispagestyle{empty}
 \markboth{\subj\ #1\hfill}{\subj\ #1\hfill}
 \pagestyle{myheadings}\examtitle{#1}{#2}
 \renewcommand{\theproblem}{Problem \arabic{problemnum}}
}

\makeatletter
\newcommand{\exambooklet}[2]{% parameters: exam name, date
 \thispagestyle{empty}
 \markboth{\subj\ #1}{\subj\ #1}
 \pagestyle{myheadings}\examtitle{#1}{#2}
 \renewcommand{\theproblem}{Problem \arabic{problemnum}}
 \renewcommand{\problem}{\newpage
 \item \let\@currentlabel=\theproblem
 \markboth{\subj\ #1, \theproblem}{\subj\ #1, \theproblem}}
}
\makeatother


\newcommand{\examtitle}[2]{\noindent\vspace*{-\toppush}\newline\parbox{6.5in}
{\textit{Analysis of Algorithms}\hfill#2\newline
CSU Channel Islands \hfill 354 Fall 2020\newline
%Singapore-MIT Alliance \hfill SMA5503\newline
\profs\hfill #1\vspace*{-.5ex}\newline
\mbox{}\hrulefill\mbox{}}\vspace*{1ex}\mbox{}\newline
\begin{center}{\Large\bf #1}\end{center}}

\newcommand{\grader}[1]{\hspace{1cm}\textsf{\textbf{#1}}\hspace{1cm}}

%\newcommand{\points}[1]{[#1 points]\ }
\newcount\pointcount
\newcommand{\points}[1]{\pointcount=#1\relax \ifnum\pointcount=1 [#1 point]\else [#1 points]\fi\ }

\newcommand{\parts}[1]
{
  \ifnum#1=1
  (1 part)
  \else
  (#1 parts)
  \fi
  \ 
}

\newcommand{\bparts}{\begin{problemparts}}
\newcommand{\eparts}{\end{problemparts}}
\newcommand{\ppart}{\problempart}

%\newcommand{\lg} {lg\ }

\setlength{\oddsidemargin}{0pt}
\setlength{\evensidemargin}{0pt}
\setlength{\textwidth}{6.5in}
\setlength{\topmargin}{0in}
\setlength{\textheight}{8.5in}


\renewcommand{\cases}[1]{\left\{ \begin{array}{ll}#1\end{array}\right.}
\newcommand{\cif}[1]{\mbox{if $#1$}}
\newcommand{\cwhen}[1]{\mbox{when $#1$}}

\newcounter{problemnum}
\newcommand{\theproblem}{Problem \theproblemsetnum-\arabic{problemnum}}
\newenvironment{problems}{
        \begin{list}{{\bf \theproblem. \hspace*{0.5em}}}
        {\setlength{\leftmargin}{0em}
         \setlength{\rightmargin}{0em}
         \setlength{\labelwidth}{0em}
         \setlength{\labelsep}{0em}
         \usecounter{problemnum}}}{\end{list}}
\makeatletter
\newcommand{\problem}[1][{}]{\item \let\@currentlabel=\theproblem \textbf{#1}}
\makeatother

\newcounter{problempartnum}[problemnum]
\newenvironment{problemparts}{
        \begin{list}{{\bf (\alph{problempartnum})}}
        {\setlength{\leftmargin}{2.5em}
         \setlength{\rightmargin}{2.5em}
         \setlength{\labelsep}{0.5em}}}{\end{list}}
\newcommand{\problempart}{\addtocounter{problempartnum}{1}\item}

\newenvironment{truefalseproblemparts}{
        \begin{list}{{\bf (\alph{problempartnum})\ \ \ T\ \ F\hfil}}
        {\setlength{\leftmargin}{4.5em}
         \setlength{\rightmargin}{2.5em}
         \setlength{\labelsep}{0.5em}
         \setlength{\labelwidth}{4.5em}}}{\end{list}}

\newcounter{exercisenum}
\newcommand{\theexercise}{Exercise \theproblemsetnum-\arabic{exercisenum}}
\newenvironment{exercises}{
        \begin{list}{{\bf \theexercise. \hspace*{0.5em}}}
        {\setlength{\leftmargin}{0em}
         \setlength{\rightmargin}{0em}
         \setlength{\labelwidth}{0em}
         \setlength{\labelsep}{0em}
        \usecounter{exercisenum}}}{\end{list}}
\makeatletter
\newcommand{\exercise}{\item \let\@currentlabel=\theexercise}
\makeatother

\newcounter{exercisepartnum}[exercisenum]
%\newcommand{\problem}[1]{\medskip\mbox{}\newline\noindent{\bf Problem #1.}\hspace*{1em}}
%\newcommand{\exercise}[1]{\medskip\mbox{}\newline\noindent{\bf Exercise #1.}\hspace*{1em}}

\newenvironment{exerciseparts}{
        \begin{list}{{\bf (\alph{exercisepartnum})}}
        {\setlength{\leftmargin}{2.5em}
         \setlength{\rightmargin}{2.5em}
         \setlength{\labelsep}{0.5em}}}{\end{list}}
\newcommand{\exercisepart}{\addtocounter{exercisepartnum}{1}\item}


% Macros to make captions print with small type and 'Figure xx' in bold.
\makeatletter
\def\fnum@figure{{\bf Figure \thefigure}}
\def\fnum@table{{\bf Table \thetable}}
\let\@mycaption\caption
%\long\def\@mycaption#1[#2]#3{\addcontentsline{\csname
%  ext@#1\endcsname}{#1}{\protect\numberline{\csname 
%  the#1\endcsname}{\ignorespaces #2}}\par
%  \begingroup
%    \@parboxrestore
%    \small
%    \@makecaption{\csname fnum@#1\endcsname}{\ignorespaces #3}\par
%  \endgroup}
%\def\mycaption{\refstepcounter\@captype \@dblarg{\@mycaption\@captype}}
%\makeatother
\let\mycaption\caption
%\newcommand{\figcaption}[1]{\mycaption[]{#1}}

\newcounter{totalcaptions}
\newcounter{totalart}

\newcommand{\figcaption}[1]{\addtocounter{totalcaptions}{1}\caption[]{#1}}

% \psfigures determines what to do for figures:
%       0 means just leave vertical space
%       1 means put a vertical rule and the figure name
%       2 means insert the PostScript version of the figure
%       3 means put the figure name flush left or right
\newcommand{\psfigures}{0}
\newcommand{\spacefigures}{\renewcommand{\psfigures}{0}}
\newcommand{\rulefigures}{\renewcommand{\psfigures}{1}}
\newcommand{\macfigures}{\renewcommand{\psfigures}{2}}
\newcommand{\namefigures}{\renewcommand{\psfigures}{3}}

\newcommand{\figpart}[1]{{\bf (#1)}\nolinebreak[2]\relax}
\newcommand{\figparts}[2]{{\bf (#1)--(#2)}\nolinebreak[2]\relax}


\macfigures     % STATE

% When calling \figspace, make sure to leave a blank line afterward!!
% \widefigspace is for figures that are more than 28pc wide.
\newlength{\halffigspace} \newlength{\wholefigspace}
\newlength{\figruleheight} \newlength{\figgap}
\newcommand{\setfiglengths}{\ifnum\psfigures=1\setlength{\figruleheight}{\hruleheight}\setlength{\figgap}{1em}\else\setlength{\figruleheight}{0pt}\setlength{\figgap}{0em}\fi}
\newcommand{\figspace}[2]{\ifnum\psfigures=0\leavefigspace{#1}\else%
\setfiglengths%
\setlength{\wholefigspace}{#1}\setlength{\halffigspace}{.5\wholefigspace}%
\rule[-\halffigspace]{\figruleheight}{\wholefigspace}\hspace{\figgap}#2\fi}
\newlength{\widefigspacewidth}
% Make \widefigspace put the figure flush right on the text page.
\newcommand{\widefigspace}[2]{
\ifnum\psfigures=0\leavefigspace{#1}\else%
\setfiglengths%
\setlength{\widefigspacewidth}{28pc}%
\addtolength{\widefigspacewidth}{-\figruleheight}%
\setlength{\wholefigspace}{#1}\setlength{\halffigspace}{.5\wholefigspace}%
\makebox[\widefigspacewidth][r]{#2\hspace{\figgap}}\rule[-\halffigspace]{\figruleheight}{\wholefigspace}\fi}
\newcommand{\leavefigspace}[1]{\setlength{\wholefigspace}{#1}\setlength{\halffigspace}{.5\wholefigspace}\rule[-\halffigspace]{0em}{\wholefigspace}}

% Commands for including figures with macpsfig.
% To use these commands, documentstyle ``macpsfig'' must be specified.
\newlength{\macfigfill}
\makeatother
\newlength{\bbx}
\newlength{\bby}
\newcommand{\macfigure}[5]{\addtocounter{totalart}{1}
\ifnum\psfigures=2%
\setlength{\bbx}{#2}\addtolength{\bbx}{#4}%
\setlength{\bby}{#3}\addtolength{\bby}{#5}%
\begin{flushleft}
\ifdim#4>28pc\setlength{\macfigfill}{#4}\addtolength{\macfigfill}{-28pc}\hspace*{-\macfigfill}\fi%
\mbox{\psfig{figure=./#1.ps,%
bbllx=#2,bblly=#3,bburx=\bbx,bbury=\bby}}
\end{flushleft}%
\else\ifdim#4>28pc\widefigspace{#5}{#1}\else\figspace{#5}{#1}\fi\fi}
\makeatletter

\newlength{\savearraycolsep}
\newcommand{\narrowarray}[1]{\setlength{\savearraycolsep}{\arraycolsep}\setlength{\arraycolsep}{#1\arraycolsep}}
\newcommand{\normalarray}{\setlength{\arraycolsep}{\savearraycolsep}}

\newcommand{\hint}{{\em Hint:\ }}

% Macros from /th/u/clr/mac.tex

\newcommand{\set}[1]{\left\{ #1 \right\}}
\newcommand{\abs}[1]{\left| #1\right|}
\newcommand{\card}[1]{\left| #1\right|}
\newcommand{\floor}[1]{\left\lfloor #1 \right\rfloor}
\newcommand{\ceil}[1]{\left\lceil #1 \right\rceil}
\newcommand{\ang}[1]{\ifmmode{\left\langle #1 \right\rangle}
   \else{$\left\langle${#1}$\right\rangle$}\fi}
        % the \if allows use outside mathmode,
        % but will swallow following space there!
\newcommand{\paren}[1]{\left( #1 \right)}
\newcommand{\bracket}[1]{\left[ #1 \right]}
\newcommand{\prob}[1]{\Pr\left\{ #1 \right\}}
\newcommand{\Var}{\mathop{\rm Var}\nolimits}
\newcommand{\expect}[1]{{\rm E}\left[ #1 \right]}
\newcommand{\expectsq}[1]{{\rm E}^2\left[ #1 \right]}
\newcommand{\variance}[1]{{\rm Var}\left[ #1 \right]}
\renewcommand{\choose}[2]{{{#1}\atopwithdelims(){#2}}}
\def\pmod#1{\allowbreak\mkern12mu({\rm mod}\,\,#1)}
\newcommand{\matx}[2]{\left(\begin{array}{*{#1}{c}}#2\end{array}\right)}
\newcommand{\Adj}{\mathop{\rm Adj}\nolimits}

\newtheorem{theorem}{Theorem}
\newtheorem{lemma}[theorem]{Lemma}
\newtheorem{corollary}[theorem]{Corollary}
\newtheorem{xample}{Example}
\newtheorem{definition}{Definition}
\newenvironment{example}{\begin{xample}\rm}{\end{xample}}
\newcommand{\proof}{\noindent{\em Proof.}\hspace{1em}}
\def\squarebox#1{\hbox to #1{\hfill\vbox to #1{\vfill}}}
\newcommand{\qedbox}{\vbox{\hrule\hbox{\vrule\squarebox{.667em}\vrule}\hrule}}
\newcommand{\qed}{\nopagebreak\mbox{}\hfill\qedbox\smallskip}
\newcommand{\eqnref}[1]{(\protect\ref{#1})}

%%\newcommand{\twodots}{\mathinner{\ldotp\ldotp}}
\newcommand{\transpose}{^{\mbox{\scriptsize \sf T}}}
\newcommand{\amortized}[1]{\widehat{#1}}

\newcommand{\punt}[1]{}

%%% command for putting definitions into boldface
% New style for defined terms, as of 2/23/88, redefined by THC.
\newcommand{\defn}[1]{{\boldmath\textit{\textbf{#1}}}}
\newcommand{\defi}[1]{{\textit{\textbf{#1\/}}}}

\newcommand{\red}{\leq_{\rm P}}
\newcommand{\lang}[1]{%
\ifmmode\mathord{\mathcode`-="702D\rm#1\mathcode`\-="2200}\else{\rm#1}\fi}

%\newcommand{\ckt}[1]{\ifmmode\mathord{\mathcode`-="702D\sc #1\mathcode`\-="2200}\else$\mathord{\mathcode`-="702D\sc #1\mathcode`\-="2200}$\fi}
\newcommand{\ckt}[1]{\ifmmode \sc #1\else$\sc #1$\fi}

%% Margin notes - use \notesfalse to turn off notes.
\setlength{\marginparwidth}{0.6in}
\reversemarginpar
\newif\ifnotes
\notestrue
\newcommand{\longnote}[1]{
  \ifnotes
    {\medskip\noindent Note: \marginpar[\hfill$\Longrightarrow$]
      {$\Longleftarrow$}{#1}\medskip}
  \fi}
\newcommand{\note}[1]{
  \ifnotes
    {\marginpar{\tiny \raggedright{#1}}}
  \fi}


\newcommand{\reals}{\mathbbm{R}}
\newcommand{\integers}{\mathbbm{Z}}
\newcommand{\naturals}{\mathbbm{N}}
\newcommand{\rationals}{\mathbbm{Q}}
\newcommand{\complex}{\mathbbm{C}}

\newcommand{\oldreals}{{\bf R}}
\newcommand{\oldintegers}{{\bf Z}}
\newcommand{\oldnaturals}{{\bf N}}
\newcommand{\oldrationals}{{\bf Q}}
\newcommand{\oldcomplex}{{\bf C}}

\newcommand{\w}{\omega}                 %% for fft chapter

\newenvironment{closeitemize}{\begin{list}
{$\bullet$}
{\setlength{\itemsep}{-0.2\baselineskip}
\setlength{\topsep}{0.2\baselineskip}
\setlength{\parskip}{0pt}}}
{\end{list}}

% These are necessary within a {problems} environment in order to restore
% the default separation between bullets and items.
\newenvironment{normalitemize}{\setlength{\labelsep}{0.5em}\begin{itemize}}
                              {\end{itemize}}
\newenvironment{normalenumerate}{\setlength{\labelsep}{0.5em}\begin{enumerate}}
                                {\end{enumerate}}

%\def\eqref#1{Equation~(\ref{eq:#1})}
%\newcommand{\eqref}[1]{Equation (\ref{eq:#1})}
\newcommand{\eqreftwo}[2]{Equations (\ref{eq:#1}) and~(\ref{eq:#2})}
\newcommand{\ineqref}[1]{Inequality~(\ref{ineq:#1})}
\newcommand{\ineqreftwo}[2]{Inequalities (\ref{ineq:#1}) and~(\ref{ineq:#2})}

\newcommand{\figref}[1]{Figure~\ref{fig:#1}}
\newcommand{\figreftwo}[2]{Figures \ref{fig:#1} and~\ref{fig:#2}}

\newcommand{\liref}[1]{line~\ref{li:#1}}
\newcommand{\Liref}[1]{Line~\ref{li:#1}}
\newcommand{\lirefs}[2]{lines \ref{li:#1}--\ref{li:#2}}
\newcommand{\Lirefs}[2]{Lines \ref{li:#1}--\ref{li:#2}}
\newcommand{\lireftwo}[2]{lines \ref{li:#1} and~\ref{li:#2}}
\newcommand{\lirefthree}[3]{lines \ref{li:#1}, \ref{li:#2}, and~\ref{li:#3}}

\newcommand{\lemlabel}[1]{\label{lem:#1}}
\newcommand{\lemref}[1]{Lemma~\ref{lem:#1}} 

\newcommand{\exref}[1]{Exercise~\ref{ex:#1}}

\newcommand{\handref}[1]{Handout~\ref{#1}}

\newcommand{\defref}[1]{Definition~\ref{def:#1}}

% (1997.8.16: Victor Luchangco)
% Modified \hlabel to only get date and to use handouts counter for number.
%   New \handout and \handoutwithouttitle commands in newmac.tex use this.
%   The date is referenced by <label>-date.
%   (Retained old definition as \hlabelold.)
%   Defined \hforcelabel to use an argument instead of the handouts counter.

\newcounter{handouts}
\setcounter{handouts}{0}

\newcommand{\hlabel}[2]{%
\stepcounter{handouts}
{\edef\next{\write\@auxout{\string\newlabel{#1}{{\arabic{handouts}}{0}}}}\next}
\write\@auxout{\string\newlabel{#1-date}{{#2}{0}}}
}

\newcommand{\hforcelabel}[3]{%          Does not step handouts counter.
\write\@auxout{\string\newlabel{#1}{{#2}{0}}}
\write\@auxout{\string\newlabel{#1-date}{{#3}{0}}}}


% less ugly underscore
% --juang, 2008 oct 05
\renewcommand{\_}{\vrule height 0 pt depth 0.4 pt width 0.5 em \,}

\renewcommand{\name}{YOUR NAME HERE}

\usepackage{amsmath}
\usepackage{url}
\usepackage{hyperref}
\usepackage{graphicx}
%\usepackage[all]{xy}
%\usepackage[ruled,lined,noend,linesnumbered]{algorithm2e}
%\SetKwIF{If}{ElseIf}{Else}{if}{}{else if}{else}{}%

\newcommand{\answer}{
 \par\medskip
 \textbf{Answer:}
}

\newcommand{\collaborators}{ \textbf{Collaborators:}
%%% COLLABORATORS START %%%
None.
%%% COLLABORATORS END %%%
}

\newcommand{\answerIa}{ \answer
%%% PROBLEM 1(a) ANSWER START %%%

%%% PROBLEM 1(a) ANSWER END %%%
}

\newcommand{\answerIb}{ \answer
%%% PROBLEM 1(b) ANSWER START %%%

%%% PROBLEM 1(b) ANSWER END %%%
}

\newcommand{\answerIc}{ \answer
%%% PROBLEM 1(c) ANSWER START %%%

%%% PROBLEM 1(c) ANSWER END %%%
}

\newcommand{\answerId}{ \answer
%%% PROBLEM 1(d) ANSWER START %%%

%%% PROBLEM 1(d) ANSWER END %%%
}

\newcommand{\answerIIa}{ \answer 
%%% PROBLEM 2(a) ANSWER START %%%

%%% PROBLEM 2(a) ANSWER END %%%
}

\newcommand{\answerIIb}{ \answer
%%% PROBLEM 2(b) ANSWER START %%%

%%% PROBLEM 2(b) ANSWER END %%%
}

\setlength{\oddsidemargin}{0pt}
\setlength{\evensidemargin}{0pt}
\setlength{\textwidth}{6.5in}
\setlength{\topmargin}{0in}
\setlength{\textheight}{8.5in}

\newcommand{\theproblemsetnum}{4}
\newcommand{\releasedate}{September 23, 2020}
\newcommand{\partaduedate}{October 2}
\newcommand{\critiqueduedate}{October 9}
\newcommand{\tabUnit}{3ex}
\newcommand{\tabT}{\hspace*{\tabUnit}}

\begin{document}

\handout{Problem Set \theproblemsetnum}{\releasedate}

\textbf{Both theory and programming questions} are due {\bf \partaduedate} at
{\bf 11:59PM}.
%
Please submit your solutions on Canvas, as two separate files: one image containing
your write-up, and one zip containing your programming solutions.

We will provide the solutions to the problem set after the problem set
is due, which you will use to find any errors in the proof that you submitted.
You will need to submit a critique of your solutions by \critiqueduedate.
Your grade will be based on both your solutions and
your critique of the solutions.

\medskip

\hrulefill

\collaborators

\begin{problems}

\problem \points{35} \textbf{Hash Functions and Load}

\begin{problemparts}

\problempart Imagine that an algorithm requires us to hash strings containing English
phrases. Knowing that strings are stored as sequences of characters, Alyssa P. Hacker
decides to simply use the sum of those character values (modulo the size of her hash
table) as the string’s hash. Will the performance of her implementation match the
expected value shown in lecture?
  \begin{enumerate}
    \item Yes, the sum operation will space strings out nicely by length.
    \item Yes, the sum operation will space strings out nicely by the characters 
      they contain.
    \item No, because reordering the words in a string will not produce a different
      hash.
    \item No, because the independence condition of the simple uniform hashing
      assumption is violated.
  \end{enumerate}
\answerIa

  \problempart Alyssa decides to implement both collision resolution and dynamic
  resizing for her hash table. However, she doesn’t want to do more work than
  necessary, so she wonders if she needs both to maintain the correctness and
  performance she expects. After all, if she has dynamic resizing, she can resize to
  avoid collisions; and if she has collision resolution, collisions don’t cause
  correctness issues. Which statement about these two properties true?
    \begin{enumerate}
      \item Dynamic resizing alone will preserve both properties.
      \item Dynamic resizing alone will preserve correctness, but not performance.
      \item Collision resolution alone will preserve performance, but not
        correctness.
      \item Both are necessary to maintain performance and correctness.
    \end{enumerate}
\answerIb
    
  \problempart Suppose that Alyssa decides to implement resizing. If Alyssa is
  enlarging a table of size $m$ into a table of size $m'$, and the table contains
  $n$ elements, what is the best time complexity she can achieve?
    \begin{enumerate}
      \item $O(m)$
      \item $O(m')$
      \item $O(n)$
      \item $O(nm')$
      \item $O(m + m')$
      \item $O(m + n)$
      \item $O(m' + n)$
    \end{enumerate}
\answerIc 

  \problempart In lecture, we discussed doubling the size of our hash table. Ivy H.
  Crimson begins to implement this approach (that is, she lets $m' = 2m$) but stops
  when it occurs to her that she might be able to avoid wasting half of the memory
  the table occupies on empty space by letting $m' = m+k$ instead, where $k$ is some
  constant. Does this work? If so, why do you think we don’t do it? There is a good
  theoretical reason as well as several additional practical concerns; a complete
  answer will touch on both points.
\answerId

\end{problemparts}

\problem \points{10} \textbf{Python Dictionaries}

We're going to get started by checking out a file from Python's Subversion repository
at svn.python.org, whose link is below. The Python project operates a web frontend
to their version control system, so we'll be able to do this using a browser.

Visit \href{http://svn.python.org/projects/python/trunk/Objects/dictnotes.txt}
{http://svn.python.org/projects/python/trunk/Objects/dictnotes.txt}.

These are notes prepared by contributors to the Python project, as they currently
exist in the Python source tree. (Cool! Actually, this document is a fascinating
read--and you should be able to understand most of it.) Read over the seven use
cases defined at the top of this document.

\begin{problemparts}

\problempart Let's examine the ``membership testing'' use case. Which statement
accurately describes this use case?
\begin{enumerate}
  \item Many insertions right after creation, and then mostly lookups.
  \item Many insertions right after creation, and then only lookups.
  \item A workload of evenly-mixed insertions/deletions and lookups.
  \item Alternating rounds of insertions/deletions and lookups.
\end{enumerate}
\answerIIa

\problempart Now imagine that you have to pick a hash function, size, collision
resolution strategy and so forth (all of the characteristics of a hash table that
we've seen so far) in order to make a hash table perfectly suited to this use case
alone. Pick the statement that best describes the choices you might make.
\begin{enumerate}
  \item A large minimum size and a growth rate of 2.
  \item A small minimum size and a growth rate of 2.
  \item A large minimum size and a growth rate of 4.
  \item A small minimum size and a growth rate of 4.
\end{enumerate}
\answerIIb

\end{problemparts}

\problem \points{55} \textbf{Matching DNA Sequences}

The code and data used in this problem are in the \texttt{dist} directory of the
homework download. Take a look at \texttt{README.txt} for some instructions.

Ben Bitdiddle has recently moved to the Kendall Square area, which is full of
biotechnology companies and their shiny, window-laden office buildings. While mocking
their dorky lab coats makes him feel slighly better about himself, he is secretly
envious, and so he sets out to earn one of his very own. To pick up the necessary
geek cred, he begins experimenting with DNA-matching technologies.

Ben would like to create mutants to do his bidding, and to get started, he'd like to
know how closesly related the creatures he's collected are. If two sequences contain
mostly the same subsequences in mostly the same places, then they're likely closely
related; if they don't, they probably aren't. (This is, of course, a gross
oversimplification.)

For our purposes, we'll represent a DNA sample as a sequence of characters. (These
characters will all be upper-case. You can look at the Wikipedia page on nucleotides
for a list of code characters and their meanings.) These sequences are very long, so
comparing subsequences of them quickly is important. We've provided code in
\texttt{kfasta.py} that reads the .fa files storing this data.

\begin{problemparts}

\problempart Let's start with \texttt{subsequenceHashes}, which returns all
length-$k$ subsequences and their hashes (and perhaps other information, if there's
anything else you might find useful). You'll want to use the \texttt{RollingHash}
implementation provided in \texttt{dnaseqlib.py}.

	Hint: There will be many subsequences;
	the DNA sequences are tens of millions of nucleotides long. To avoid keeping them
	all in memory at once, implement your function as a generator. See the Python
	reference materials available online (or search ``Python generators'') if you
	aren't familiar with this construct.

\problempart Implement \texttt{Multidict} and verify that your work passes the simple
sanity tests provided.

	\texttt{Multidict} should behave just like a Python dictionary, except that it can
	store multiple values per key. If no values exist for a key, it returns an empty
	list; otherwise, it returns the list of associated values. You may (and probably
	should) use the Python dictionary in your implementation.

\problempart Now it's time to implement \texttt{getExactSubmatches}. Again, 
implementing this function as a generator is probably a good idea--you will have
many, many matches. Much of the work has already been done by \texttt{Multidict}
and \texttt{subsequenceHashes}; with these completed, the implementation of
\texttt{getExactSubmatches} does not need to be
very complex (my implementation is 9 lines long).

	This function should return pairs of offsets into the inputs. A tuple $(x, y)$
	being returned indicates that the $k$-length subsequence starting at index $x$
	of the first input matches the one starting at index $y$ of the second input.

	A simple sanity test has been provided in \texttt{test\_dnaseq.py}.
	\texttt{README.txt} describes how to run this test.

	In the next part, you will run tests that take around 30 minutes each.
	If you want to test on shorter runs first, then test using the shorter
	sequences provided in the \texttt{data} folder. Follow instructions in
	\texttt{README.txt} to run said tests.

\problempart Run comparisons between the two human samples (maternal and paternal)
and between the paternal sample and the chimp sample, and compare the outputs to
the expected outputs. These tests will take 15-30 minutes each to run.

	Feel free to peak at the image-generation code in \texttt{libdnaseq.py} to see
	how it works. What it's doing is keeping track of how many of our $(x, y)$
	matches land in each bin in a two-dimensional grid of bins, each of which
	corresponds to a pixel in the output image. At the end, it normalizes the
	match counts in each bin, so the largest number of matches is black and zero
	matches is white.

	What should a perfect match (i.e. comparing a sequence to itself) look like?
	Hint: if $x=y$, then $(x, y)$ should be a match. Test your results by checking
	the output when you compare one of the data files to itself.

	These tests take a while with $O(1)$ lookup time... Imagine how long they would
	take if this was implemented with, say, a binary tree with logarithmic lookup
	time! There are tens of millions (i.e. about $2^{25}$) elements in each sequence,
	so runs would take somewhere on the scale of 12 hours! This is, of course,
	likely a very bad estimate, as it does not account for any of the constant
	factor differences between the implementations.

\end{problemparts}

When you're done, submit your writeup and your \texttt{dnaseq.py} as two separate
files on Canvas (do not zip them up and submit them together).

\end{problems}
\end{document}
